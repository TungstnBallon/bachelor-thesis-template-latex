\chapter{Evaluation}
\label{chapter:Evaluation}

The focus of the evaluation will be on the different runtimes of the various possible backends. These will be evaluated by running varouis prepared Jayvee models on one underlying dataset.

\section{Data source}
\label{section:data_source}

\subsection{Criteria}
\label{subsection:data_source_criteria}
The dataset has to be available in a csv format, because the interpreter can only create tables from csv data.
Jayvee tables can only contain boolean, text, integer and decimal values, so data other than that (e.g. dates) will not be parsed and represented as text.
The polars backend cannot transform dates, so the dataset must contain numbers or booleans.
The dataset should be a minimum of $800\text{MB}$, to make the backend's differences in speed visible.
The dataset should not be larger than $3\text{GB}$, because open-datasets are usually not that big. %FIXME: citation needed +  everything.
The dataset must have an open license.% TODO: explanation.

The chosen dataset is called "Brewery Operations and Market Analysis" and not based on any real wold data, but rather generated by a python script \autocite{dataset}.




\section{Parameters}
\label{section:parameters}

\section{Jayvee limitations}
\label{section:jv_limits}

During the evaluation phase we encountered two (current) limitations of the jayvee interpreter limiting % FIXME: repetition
the ammout of data that can be processed. The first limitation % FIXME: again
is that, because jayvee runs inside `node.js` %FIXME
, it cannot load files bigger than $2\text{GB}$ fully into memory.
However, since many of the open datasets are smaller than that, %FIXME: citation needed
we chose to accept this limitation.

The second interpreter limitation also derives from it running inside of node.
The `TextFileInterpeter` %FIXME: code
block processes binary data into a list of lines for other blocks to consume
