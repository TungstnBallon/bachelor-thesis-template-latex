\chapter{Background}
\label{chapter:Literature}

At the time of writing jayvee only supports csv as a data source and csv, sqlite and postgres as a data sink, all of which utilize tables as a central abstraction over their dat




\section{Columnar vs. Rows}
\label{section:column_vs_row}
How Jayvee lays out data internaly is a crucial factor in metrics like execution speed and memory efficency.
Many formats for this exist, but all of them are either row or column oriented.

The row based format saves the values of a row sequentially.

The columnar format saves the values of a column seuantially.

\section{Columnar Formats}
\label{section:columnar}
Starting around 2011, multiple proposals for a columnar data layour began appearing, resulting in the creation of Apache Parquet and Apache ORC, both disk-based columnar formats.

\section{Apache Arrow}
\label{section:arrow}
"The Apache Arrow project was initiated by the Apache Foundation in 2016.
This framework provides an open and a common standardized format for different programming languages for reading/writing tabular data in-memory.
Through language-specific libraries, multiple languages can share data without any copying or serialization." \autocite{Ahmad2020} (see \ref{fig:arrow_com})
"In the Arrow format, data entries (records) are stored in a table called a RecordBatch.
Each record field is stored in a separate column of the RecordBatch table in a manner that is as contiguous as possible in memory.
This is called an Arrow Array which can store data of different types—i.e., int, float, strings, binary, timestamps and lists, but also nested types (such as lists of lists, etc.).
Arrays may have different types of physical buffers to store data.
This layout provides higher spatial locality when iterating over column contiguous data entries for better CPU cache performance.
SIMD (Single instruction, multiple data) vector operations can also benefit from such a layout as vector elements are already aligned properly in memory." \autocite{Ahmad2020}

\begin{figure}
	\includegraphics[width=\textwidth]{resources/arrow_tab}
	\caption{Key Concepts of an Arrow Table \autocite{Dremio}}
	\label{fig:arrow_tab}
\end{figure}
\begin{figure}
	\includegraphics[width=\textwidth]{resources/arrow_interop}
	\caption{Example how Arrow increases interoperability between proceses \autocite{Ahmad2020}}
	\label{fig:arrow_com}
\end{figure}




\section{Polars}
\label{section:polars}
Polars not only implements the apache arrow format, but adds convenient abstractions like dataframes and series and promises a drastic performance increase.
