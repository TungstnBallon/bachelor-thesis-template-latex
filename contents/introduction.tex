\chapter{Introduction}
\label{chapter:Introduction}

Each year the European Union publishes a report on the maturity of open data offerings by its member states.
The European Union's anual Open Data Maturity Report shows a $19\%$ increase in average maturity scores since 2018. % TODO: citation
The use of this open data is diminished by the often significant technical knowledge required to work with it.
Lowering this barrier of entry is the major motivation beind the JValue Project \autocite{jvalue:landing} in general and its language, Jayvee, specifically \autocite{jvalue:jayvee}.

\section{The Jayvee language}
\label{section:jv_intro}

Jayvee's goal is to allow everyone to describe ETL pipelines \autocite{jvalue:jayvee}.
Pipelines are made up of a user defined sequence of blocks, which fall into one of three categories: Extractor, Tranformer or Loader.

Extractors bring the data into the pipeline and make it available for the other types of blocks.
They can retrieve data from local files (\Verb|LocalFileExtractor|) and urls (\Verb|HttpExtractor|).

Loaders export data to somewhere outside the pipeline.
Most common is an sqlite file (\Verb|SQLiteLoader|), but postgres (\Verb|PostgresLoader|) and csv (\Verb|CSVFileLoader|) are also supported.

Between an extractor and a loader the user can define an arbitrary amount of transformer blocks.
They have one input and one output, both with a specific type like \Verb|TextFile| \Verb|Table|.
Some transformers change the type of the data, like \Verb|CSVInterpreter| which interprets a text file as a csv sheet or \Verb|TableInterpreter|, who converts a sheet into a table.
Others transform their input data, like \Verb|CellWriter| or \Verb|TableTranformer|.
To accomplish this they use transforms, which are similar to a map function in other programming languages.
They let the user define multiple inputs with their types (jayvee supports \Verb{text}, \Verb{integer}, \Verb{decimal} and \Verb{boolean}), an output with its type and an expression to transform the inputs into the output.


Jayvee uses transforms to compute new values from a sequence of defined inputs, analogous to what a function would to in other programming languages \autocite{jvalue:jayvee:docs:transfrom}.
The concrete operation, who's result a transform computes, is described by an expression, e.g \Verb|x + 3|.

Expressions support basic arithmetic, boolean algebra and rudimentary string operations like \Verb|uppercase| or \Verb|matches|.

%TODO: picture
\begin{figure}
	\begin{plantuml}
		@startuml
		start
		:LocalFileExtractor;
		->File;
		:TextFileInterpreter;
		->TextFile;
		:CSVFileInterpreter;
		->Sheet;
		:TableInterpreter;
		->Table;
		:TableTransformer;
		->Table;
		:SQLiteLoader;
		stop
		@enduml
	\end{plantuml}
	\caption{The Flow of an examplory pipeline in jayvee. Between the blocks is noted the type of the data called IOType}
	\label{fig:}
\end{figure}


Optimizing how jayvee represents data in-memory is the goal of this thesis.

\newpage

{\textsl{General instructions:}}
You should make use of our outline for structuring your
thesis. The specific outline is determined by your thesis type.
Please refer to the outlines at
\url{https://oss.cs.fau.de/theses/structure-content/}.

{\textsl{\LaTeX\ instructions:}} Here is an example citation
\autocite{riehle:2011:controlling}.
\textcite{riehle:2007:economic} gave another example citation.

\begin{figure}[ht]
	\includegraphics[width=0.5\textwidth]{resources/\logo}
	\caption{Example of a Figure}
	\label{fig:example}
\end{figure}

For more convenient creation of latex tables we recommend:

\url{https://www.tablesgenerator.com/}

\begin{table}[ht]
	\caption{Example of a Table}
	\label{tab:example}
	\begin{tabular}{|l|l|}
		\hline
		Nice column      & Very nice column          \\
		\hline
		some content     & some more content         \\
		some information & some ... (you guess what) \\
		\hline
	\end{tabular}
\end{table}

I am using an the \ac{FACR} here. The second usage of \ac{FACR} will only display the acronym itself.
