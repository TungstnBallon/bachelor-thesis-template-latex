\chapter{Introduction}
\label{chapter:Introduction}

Each year the European Union publishes a report on the maturity of open data offerings by its member states.
The European Union's anual Open Data Maturity Report shows a $19\%$ increase in average maturity scores since 2018. % TODO: citation
The use of this open data is diminished by the often significant technical knowledge required to work with it.
Lowering this barrier of entry is the major motivation beind the JValue Project \autocite{jvalue:landing} in general and its language, Jayvee, specifically \autocite{jvalue:jayvee}.

\section{The Jayvee language}
\label{section:jv_intro}

A pipeline is a sequence of executable blocks to perform operations on data.
These blocks are categorized in extractors, transformers and loaders.

\begin{figure}[h]
	\begin{center}
		\begin{plantuml}
			@startuml
			(*) -> "Extractor"
			-> "Transformer"
			-> "Loader"
			-> (*)
			@enduml
		\end{plantuml}
	\end{center}
\end{figure}

\textbf{Extractors} bring the data into the pipeline and make it available for the other types of blocks.
Jayvee supports retrieving data from local files (\Verb|LocalFileExtractor|) and \ac{URL}s (\Verb|HttpExtractor|).

\textbf{Transformers} modify the data between extractor and loader.

\textbf{Loaders} export data to somewhere outside the pipeline.
Most common is an sqlite file (\Verb|SQLiteLoader|), but postgres (\Verb|PostgresLoader|) and csv (\Verb|CSVFileLoader|) are also supported.

The Jayvee interface of blocks (see %TODO: reference architecture chapter
) defines one input and one output port.
The output port can be linked to multiple input ports.
To be clarified: multplle output ports to one input. %TODO:

A more detailed description of Jayvee's core concepts has been provided by \textcite{jvalue:jayvee:docs:core_concepts}.

Jayvee's goal is to simply allow everyone to describe ETL pipelines \autocite{jvalue:jayvee}.

\newpage

{\textsl{General instructions:}}
You should make use of our outline for structuring your
thesis. The specific outline is determined by your thesis type.
Please refer to the outlines at
\url{https://oss.cs.fau.de/theses/structure-content/}.

{\textsl{\LaTeX\ instructions:}} Here is an example citation
\autocite{riehle:2011:controlling}.
\textcite{riehle:2007:economic} gave another example citation.

\begin{figure}[ht]
	\includegraphics[width=0.5\textwidth]{resources/\logo}
	\caption{Example of a Figure}
	\label{fig:example}
\end{figure}

For more convenient creation of latex tables we recommend:

\url{https://www.tablesgenerator.com/}

\begin{table}[ht]
	\caption{Example of a Table}
	\label{tab:example}
	\begin{tabular}{|l|l|}
		\hline
		Nice column      & Very nice column          \\
		\hline
		some content     & some more content         \\
		some information & some ... (you guess what) \\
		\hline
	\end{tabular}
\end{table}
