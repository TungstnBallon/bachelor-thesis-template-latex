\chapter{Architecture}
\label{chapter:Architecture}

\section{Requirements}
\label{section:Requirements}

We follow the established practice of differentiating between functional and non functional requirements.

\subsection{Functional Requirements}
\label{subsection:FunctionalRequirements}

Tables are represented with a standardized, columnar, in-memory format.
No blocks become unsupported because of this transition.

\subsection{Non Functional Requirements}
\label{subsection:NonFunctionalRequirements}

The time needed to execute \mintinline{typescript}{.jv} files decreases.
The source code is readable for non-rust developers.
Informative log messages are generated.
The source code adheres to the project's code style.

\section{Previous Situation}
\label{section:prev}
Pipeline definitions in jayvee consist of a series of block definitions.
At runtime, the jayvee interpreter associates each block-definition with a class implementing the \mintinline{typescript}|BlockExecutor<I extends IOType, O extends IOType>| interface.
\mintinline{typescript}|IOType| is an enum specifing all of the allowed types inputs and ouptuts can have (see \ref{lst:iotype}), which is made use of by the \mintinline{typescript}|IOTypeImplementation<T extends IOType>| interface, whose implementors are the actual inputs and ouptuts.
\begin{listing}
	\begin{minted}{typescript}
enum IOType {
	NONE = 'None',
	FILE = 'File',
	TEXT_FILE = 'TextFile',
	FILE_SYSTEM = 'FileSystem',
	SHEET = 'Sheet',
	TABLE = 'Table',
	WORKBOOK = 'Workbook',
}
\end{minted}
	\caption{The \mintinline{typescript}{IOType} enum}
	\label{lst:iotype}
\end{listing}
One of these is the \mintinline{typescript}|Table| class which implements \mintinline{typescript}|IOTypeImplementation<IOType.TABLE>|, and can thus serve as both input and output to blocks using tables.
For an overview of how \mintinline{typescript}{Table} fits into the interpreter, see \ref{fig:prev_sit}.
% This interface also provides the method \mint{typescript}|execute(input: IOTypeImplementation<I>, context: ExecutionContext): Promise<R.Result<IOTypeImplementation<O>>>|.
% These inputs/outputs must fall into one of seven categories (see \ref{lst:iotype}).
% Jayvee uses this \mintinline{typescript}|IOType| via \mintinline{typescript}|IOTypeImplementation<T>| where \mintinline{typescript}|T| is a variant of the \mintinline{typescript}|IOType| enum.
\begin{figure}
	\begin{plantuml}
		@startuml
		hide empty members

		interface PipelineDefinition
		interface BlockDefinition

		interface BlockExecutor<I extends IOType, O extends IOType>

		interface BlockExecutorClass

		PipelineDefinition "1" *-- "many" BlockDefinition
		BlockDefinition "1" --> "1" BlockExecutor: interpreter finds

		interface IOTypeImplementation<T extends IOType>
		IOTypeImplementation - BlockExecutor: input
		IOTypeImplementation - BlockExecutor: output

		class Table
		IOTypeImplementation <|.. Table : IOTypeImplementation<IOType.TABLE>

		abstract class AbstractBlockExecutor<I extends IOType, O extend IOType> implements BlockExecutor

		class TableTransformerExecutor {}
		AbstractBlockExecutor <|-- TableTransformerExecutor: AbstractBlockExecutor<IOType.TABLE, IOType.TABLE>
		BlockExecutorClass <|-- TableTransformerExecutor
		@enduml
	\end{plantuml}
	\caption{How \mintinline{typescript}|Table| fits into the interpreter} %FIXME
	\label{fig:prev_sit}
\end{figure}

Looking at the code in \ref{lst:prev_table}, we can see that \mintinline{typescript}|Table| already follows a columnar format.
\begin{listing}
	\begin{minted}[linenos, firstnumber=35]{typescript}
export class Table implements IOTypeImplementation<IOType.TABLE> {
  public readonly ioType = IOType.TABLE;

  private numberOfRows = 0;

  private columns = new Map<string, TableColumn>();
}
	\end{minted}
	\caption{\mintinline{typescript}{libs/execution/src/lib/types/io-types/table.ts}} %FIXME
	\label{lst:prev_table}
\end{listing}

\section{General Idea}
\label{section:General Idea}

\subsection{Apache Arrow}
\label{subsection:apache arrow}
Adapting Jayvee Tables to implement the apache arrow format fulfills the functional requirements of a standardized, columnar, in-memory representation.
Additionaly Arrow is optimized for modern CPUs and GPUs, and organized for efficent analytical workloads, which will get us closer to the non functional requrement of decreased runtime.





The first decision is to create a new class \mintinline{typescript}|PolarsTable|, which utilises from \mintinline{typescript}|polars.DataFrame| under the hood %FIXME
. This follows the adapter pattern.
% We decided the best approach to the requirements outlined in \ref{section:Requirements} is to place one layer of abstraction between the current \mintinline{typescript}|Table| class and the systems it interacts with.
As described in \ref{section:prev}, if a class wants to be allowed as an input or output, it must implement the \mintinline{typescript}{IOTypeImplementation<T>} interface where \mintinline{typescript}{T} is one of the variants of \ref{lst:iotype}.
However, we decided against implementing \mintinline{typescript}{IOTypeImplementation<T>} seperately for \mintinline{typescript}|PolarsTable| and \mintinline{typescript}|Table| and chose to make \mintinline{typescript}|Table| an abstract class, which meant moving the existing functionality into a new class \mintinline{typescript}|TsTable|.
% FIXME: Table vs TsTable is confusing
\begin{figure}
	\begin{plantuml}
		@startuml
		hide empty members
		interface IOTypeImplementation<T extends IOType = IOType> {
				+ioType: T
				+acceptVisitor<R>(vistor: IoTypeVisitor<R>): R
			}
		abstract class Table {
				+ioType: IOType.TABLE
			}
		IOTypeImplementation <|.. Table : IOtypeImplementation<IOType.TABLE>
		abstract class TableColumn {
			}
		class PolarsTable extends Table {
				-df: polars.DataFrame
			}
		class TsTable extends Table {
				+columns: Map<string, TsTableColumn>
			}
		class PolarsTableColumn extends TableColumn {
				-series: polars.Series
			}
		class TsTableColumn<T extends InternalValueRepresentation> extends TableColumn {
			}
		TsTable "1" *-- "many" TsTableColumn
		PolarsTable "1" *-- "1" polars.DataFrame: df
		PolarsTableColumn "1" *-- "1" polars.Series: series
		@enduml
	\end{plantuml}
	\caption{How jayvee represents tables} % FIXME
	\label{fig:current_sit}
\end{figure}


\subsection{Implementing the adapter pattern}
\label{subsection:adapter}

\mintinline{typescript}|PolarsTable| serves as an adapter between \mintinline{typescript}|Table| and \mintinline{typescript}|polars.DataFrame|
