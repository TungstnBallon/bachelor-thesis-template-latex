\chapter{Architecture}
\label{chapter:Architecture}

\section{Requirements}
\label{section:Requirements}

We follow the established practice of differentiating between functional and non functional requirements.
These requirements have been evaluated for completenes, consistency and feasability.

\subsection{Functional Requirements}
\label{subsection:FunctionalRequirements}

% As we've described in \ref{chapter:Literature} tables are a crucial factor of the Jayvee interpreter's performance.
Tables are the central way Jayvee represent's data.
The other types like \Verb|TextFile| or \Verb|Sheet| mostly exists to be parsed into tables.
Hence % TODO: ?
we shall focus on how to efficiently represent tables in memory.
Since Jayvee's \Verb|TableTransformer| work on columns and not rows, the advantages of the columnar format outlined in %TODO: ref
lend themselves well to jayvee.
Jayvee needs manipulate the data, so data needs to be stored in memory not on disk.
This leads to the requirement that \emph{tables are represented with a columnar, in-memory data layout}.

The goal of this thesis is not to create a fully mature implementation of this new format, but rather to prototype an integration and evaluate the results of this migration.
However to accurately evaluate the changes, the core functionality of jayvee (parse data into table, apply function to columns of that table, export table) needs to be preserved.
\emph{Core functionality is maintained}.
This includes that tables can still be created in the same ways (\Verb|TableIntepreter|) and exported with the same blocks (\Verb|SQLiteLoader|) (\Verb|PostresLoader|).
\emph{Arithmetic operations work in transforms}.

Should it be necessary to write code in a language other than typescript, this code must be placed in an external library usable from typesript.
This ensures usability and increases maintainability by for other developers.

To make evaluation easier the new implementation must be choosable with a commandline flag \emph{at runtime}.

\subsection{Non Functional Requirements}
\label{subsection:NonFunctionalRequirements}

The main goal of this thesis is to \emph{optimize} Jayvee's in-memory representation.
Therefore, the \emph{time needed to execute Jayvee pipelines should decrease}.

The source code is readable for non-rust developers.
Informative log messages are generated.
The source code adheres to the project's code style.

\section{Interpreter Overview}
\label{section:prev}

In this section, we will give an overview of how the interpreter executes a jayvee pipeline.

Pipeline definitions in jayvee consist of a series of block definitions.
At runtime, the jayvee interpreter associates each block-definition with a class implementing the \mintinline{typescript}|BlockExecutor<I extends IOType, O extends IOType>| interface.
\mintinline{typescript}|IOType| is an enum specifing all of the allowed types inputs and ouptuts can have (see \ref{lst:iotype}), which is made use of by the \mintinline{typescript}|IOTypeImplementation<T extends IOType>| interface, whose implementors are the actual inputs and ouptuts.
\begin{listing}
	\begin{minted}{typescript}
enum IOType {
	NONE = 'None',
	FILE = 'File',
	TEXT_FILE = 'TextFile',
	FILE_SYSTEM = 'FileSystem',
	SHEET = 'Sheet',
	TABLE = 'Table',
	WORKBOOK = 'Workbook',
}
\end{minted}
	\caption{The \mintinline{typescript}{IOType} enum}
	\label{lst:iotype}
\end{listing}
One of these is the \mintinline{typescript}|Table| class which implements \mintinline{typescript}|IOTypeImplementation<IOType.TABLE>|, and can thus serve as both input and output to blocks using tables.
For an overview of how \mintinline{typescript}{Table} fits into the interpreter, see \ref{fig:prev_sit}.
% This interface also provides the method \mint{typescript}|execute(input: IOTypeImplementation<I>, context: ExecutionContext): Promise<R.Result<IOTypeImplementation<O>>>|.
% These inputs/outputs must fall into one of seven categories (see \ref{lst:iotype}).
% Jayvee uses this \mintinline{typescript}|IOType| via \mintinline{typescript}|IOTypeImplementation<T>| where \mintinline{typescript}|T| is a variant of the \mintinline{typescript}|IOType| enum.
\begin{figure}
	\begin{plantuml}
		@startuml
		hide empty members

		interface PipelineDefinition
		interface BlockDefinition

		interface BlockExecutor<I extends IOType, O extends IOType> {
				+execute(input: IOTypeImplementation<I\>, context: ExecutionContext): Result<IOTypeImplementation<O>>
			}

		interface BlockExecutorClass {}

		PipelineDefinition "1" *-- "many" BlockDefinition
		BlockDefinition "1" --> "1" BlockExecutor: interpreter finds

		interface IOTypeImplementation<T extends IOType>
		IOTypeImplementation - BlockExecutor: input
		IOTypeImplementation - BlockExecutor: output

		class Table
		IOTypeImplementation <|.. Table : IOTypeImplementation<IOType.TABLE>

		abstract class AbstractBlockExecutor<I extends IOType, O extends IOType> implements BlockExecutor {
		+ {abstract} doExecute(input: IOTypeImplementation<I\>, context: ExecutionContext): Result<IOTypeImplementation<O\>>
		}

		class TableTransformerExecutor {
				+doExecute(input: Table, context: ExecutionContext): Result<Table>
			}
		AbstractBlockExecutor <|-- TableTransformerExecutor: AbstractBlockExecutor<IOType.TABLE, IOType.TABLE>
		BlockExecutorClass <|-- TableTransformerExecutor
		@enduml
	\end{plantuml}
	\caption{How \mintinline{typescript}|Table| fits into the interpreter} %FIXME
	\label{fig:prev_sit}
\end{figure}

Looking at the code in \ref{lst:prev_table}, we can see that \mintinline{typescript}|Table| already follows a columnar format.
\begin{listing}
	\begin{minted}[linenos, firstnumber=35]{typescript}
export class Table implements IOTypeImplementation<IOType.TABLE> {
  public readonly ioType = IOType.TABLE;

  private numberOfRows = 0;

  private columns = new Map<string, TableColumn>();
}
	\end{minted}
	\caption{\mintinline{typescript}{libs/execution/src/lib/types/io-types/table.ts}} %FIXME
	\label{lst:prev_table}
\end{listing}

\section{General Idea}
\label{section:General Idea}

Adapting Jayvee Tables to implement the apache arrow format fulfills the functional requirements of a standardized, columnar, in-memory representation.
Additionaly Arrow is optimized for modern CPUs and GPUs, and organized for efficent analytical workloads, which will get us closer to the non functional requrement of decreased runtime. % TODO: citation

\begin{figure}
	\begin{plantuml}
		@startuml
		start
		:start interpreter;
		:parse model;
		if (parseOnly) then (yes)
		else (no)
		while (more pipeline)
		partition run pipelines {
				:sort blocks into execution order;
				while (more blocks)
				partition run block {
						partition get block exectutor {
								if (block has multiple executors) then (yes)
								if (usePolars) then (true)
								:pick polars executor;
								else (false)
								:pick typescript executor;
								endif
								else (no)
								:pick general executor;
								endif
							}
						:call exectutor;
						note
						<code>
						BlockExecutor<I extends IOType,O extends IOType>.execute(
						input: IOTypeImplementation<I>,
						context: ExectutionContext
						): Result<IOTypeImplementation<O>>
						</code>
						end note
						:save output as input for the next block;
					}
				endwhile
			}
		endwhile
		endif
		stop
		@enduml
	\end{plantuml}
	\caption{An activity diagram overview of the interpreter}
	\label{fig:interpreter_activity}
\end{figure}


\subsection{Possible implementations}
\label{subsection:arrow implementations}

In this section, we will discuss possible implementations of the arrow specification.

\subsubsection{Implementation from scratch}
\label{subsubsection:direct_implementation}
The most direct option is to implement the Arrow specification directily.
This would have the advantage of having full control over the implementation.
It is not guaranteed that an enough of the specification can be implemented within the timeframe of the thesis.
This implementation would probably perform worse than a mature implementation.
Correctness is hard to guarantee.

An implementation from scratch is \emph{too big in scope} for a bachelor's thesis.

\subsubsection{Apache arrow in JS}
\label{subsubsection:arrow_in_js}
There is an implementation of arrow in typescript.
This has the advantage of being the same lanugage as the jayvee interpreter.
typescript on nodejs has a reputation of being slow compared to other languages. % TODO: citation
this implementation is comparable to other languages in terms of supported datatypes, but is lacks behind in support for other apache frameworks \autocite{arrow:status}, which takes away one of arrow's strengths.

Considering this, an implementation inside a compiled language, that offers a typescript api seems preferable.

\subsubsection{Polars}
\label{subsection:polars}
As described in % TODO: reference.
Polars is a high level api built on top of apache arrows, but it also offers useful abstractions like dataframes.
It also promises huge %FIXME: wording
speed gains compared to it's competitors %TODO: citation

%TODO: bild über abstraktion polars zu js.

The Dataframe abstractions allows us to get the in-memory layout we want, by simply creating an adapter %FIXME: design pattern is strategy not adapter
between the interpeter classes utilising tables and the polars dataframe.


\section{The strategy pattern}
The requirement to pick the implementation at runtime makes the strategy pattern really appealing. %FIXME: wording
%TODO: Describe strategy pattern more

\section{new table} %TODO: find an actual title
\label{section:new_table}


The Polars class \Verb|DataFrame| is made useful to the jayvee interpreter by wrapping it in a new class \Verb|PolarsTable|.

We move all of the functionality from \Verb{Table} to \Verb{TsTable} and create a new \emph{abstract} class \Verb|Table|.
This differs from the usual stratergy pattern in that, normally, an interface is used to represent the shared functionality between the algorithms.
We feel that an abstract class represents the relationship between \Verb|TsTable| and \Verb|PolarsTable| better than an interface. %TODO: why?

The first decision is to create a new class \mintinline{typescript}|PolarsTable|, which utilises from \mintinline{typescript}|polars.DataFrame| under the hood %FIXME
.
% We decided the best approach to the requirements outlined in \ref{section:Requirements} is to place one layer of abstraction between the current \mintinline{typescript}|Table| class and the systems it interacts with.
As described in \ref{section:prev}, if a class wants to be allowed as an input or output, it must implement the \mintinline{typescript}{IOTypeImplementation<T>} interface where \mintinline{typescript}{T} is one of the variants of \ref{lst:iotype}.
However, we decided against implementing \mintinline{typescript}{IOTypeImplementation<T>} seperately for \mintinline{typescript}|PolarsTable| and \mintinline{typescript}|Table| and chose to make \mintinline{typescript}|Table| an abstract class, which meant moving the existing functionality into a new class \mintinline{typescript}|TsTable|.
% FIXME: Table vs TsTable is confusing
\begin{figure}
	\begin{plantuml}
		@startuml
		hide empty members
		interface IOTypeImplementation<T extends IOType = IOType> {
				+ioType: T
				+acceptVisitor<R>(vistor: IoTypeVisitor<R>): R
			}
		abstract class Table {
				+ioType: IOType.TABLE
			}
		IOTypeImplementation <|.. Table : IOtypeImplementation<IOType.TABLE>
		abstract class TableColumn {
			}
		class PolarsTable extends Table {
				-df: polars.DataFrame
			}
		class TsTable extends Table {
				+columns: Map<string, TsTableColumn>
			}
		class PolarsTableColumn extends TableColumn {
				-series: polars.Series
			}
		class TsTableColumn<T extends InternalValueRepresentation> extends TableColumn {
			}
		TsTable "1" *-- "many" TsTableColumn
		PolarsTable "1" *-- "1" polars.DataFrame: df
		PolarsTableColumn "1" *-- "1" polars.Series: series
		@enduml
	\end{plantuml}
	\caption{How jayvee represents tables} % FIXME
	\label{fig:current_sit}
\end{figure}

\section{new executors}
